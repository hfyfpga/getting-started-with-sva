\documentclass{beamer}

\usetheme{Warsaw}
\title{Getting started with SystemVerilog Assertions}
\author{Tudor Timi}
\institute{VerificationGentleman.com}
\date{April 15, 2020}
\begin{document}


\begin{frame}[fragile]{Cover property protips}

\begin{block}{Protip}
Cover antecedents
\end{block}

\begin{semiverbatim}
assert property (req ##1 grant [->1] |=> busy);

cover property (req ##1 grant [->1);
\end{semiverbatim}

Cover property could be generated automatically by the tool
\end{frame}


\begin{frame}[fragile]{Cover property protips}
\begin{block}{Protip}
Avoid covering an implication property
\end{block}

\begin{semiverbatim}
cover property (req ##1 grant [->1] |=> busy);
\end{semiverbatim}

Tool will most likely show an empty trace
\end{frame}

\begin{frame}[fragile]{Cover property protips}
\begin{block}{Protip}
Use cover properties to get acquainted with a design
\end{block}

Most relevant for verification engineers and for designers taking over a new design
\end{frame}


\begin{frame}[fragile]{Cover property protips}
\begin{block}{Protip}
Use cover properties inside generate statements to ensure all modes are reachable
\end{block}

\begin{semiverbatim}
// Not 100% legal SV code
for (req_kind_e kind = kind.first();
    kind <= kind.last(); kind++)
begin: all_req_kinds
  cover property (
      req && req_kind == kind ##1 grant [->1]);
end
\end{semiverbatim}

Emulates a covergroup sampled when the sequence matches. Covergroups could be supported by the tool
\end{frame}


\end{document}